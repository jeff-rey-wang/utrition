\documentclass{article}

\usepackage{tabularx}
\usepackage{booktabs}

\title{Reflection Report on \progname}

\author{\authname}

\date{}

\input{../Comments}
%% Common Parts

\newcommand{\progname}{Utrition} % PUT YOUR PROGRAM NAME HERE
\newcommand{\authname}{Team 16, Durum Wheat Semolina
	\\ Alexander Moica
	\\ Yasmine Jolly
	\\ Jeffrey Wang
	\\ Jack Theriault
	\\ Catherine Chen
	\\ Justina Srebrnjak } % AUTHOR NAMES                 

\usepackage{hyperref}
    \hypersetup{colorlinks=true, linkcolor=blue, citecolor=blue, filecolor=blue,
                urlcolor=blue, unicode=false}
    \urlstyle{same}
                                


\begin{document}

\begin{table}[hp]
\caption{Revision History} \label{TblRevisionHistory}
\begin{tabularx}{\textwidth}{llX}
\toprule
\textbf{Date} & \textbf{Developer(s)} & \textbf{Change}\\
\midrule
April 5, 2023 & All & Initial Document\\
\bottomrule
\end{tabularx}
\end{table}

\newpage

\maketitle

This document summarizes the overall experience of the creation of the Utrition application. 

\section{Project Overview}

\plt{Summarize the original project goals and requirements}

The idea behind Utrition was to create an application that was ideal for the student lifestyle. We wanted our users to be able to easily keep track of their foods and view their eating trends. In order to accomplish this, Utrition would log all foods that the user has eaten. Centering our application's design of usability, we developed multiple different manners of uploading a food item such as a voice upload, a text upload, and an image upload. Based on this logged information and additional personal physical statistics that the user would input, Utrition would total how many calories the user has eaten per day and tell them how many calories they would need to consume in order to maintain their weight/body mass index. In addition to this we had requirements of reflecting back all of the foods the user logged with a date stamp so they can keep track of what they have eaten every day. All of these requirements combined to make the final Utrition product, a user-friendly application designed for a busy lifestyle.


\section{Key Accomplishments}

\plt{What went well?  This can be what went well with the documentation, the
  coding, the project management, etc.}

Various aspects of the development of Utrition can be viewed as a success or a key accomplishment. The final design was derived from various levels of user testing. This user testing was key to our success since it allowed us to create a product that users would actually enjoy using. These improvements were validated as we did additional testing after our revisions and saw improvements in the ratings of how satisfied our users were. Additionally, we had a very effective and organized workflow. As a team, we were able to stay on top of deadlines by employing the Git project board. By using this tool, we were able to evenly distribute work and resolve merging conflicts very easily. We also created a weekly meeting schedule in order to keep track of everyone's progress. In these meetings, we would discuss our partially completed work, provide next steps and internal deadlines, and voice any issues that are currently blocking a developer's progress. 

\section{Key Problem Areas}

\plt{What went wrong?  This can be what went wrong with the documentation, the
  technology, the coding, time management, etc.}

  As our goal was to create a user-friendly application, a couple of changes were made in response to feedback from our instructor and our user testing. We initially planned for our application to have a heavy machine learning aspect. In the first stages of Utrition, there was originally only the ability to upload an image. Due to training data availability restrictions, the artificial intelligence we had built was only trained to be able to correctly identify 5 foods. We had not done enough initial research and had to heavily change our requirements in the middle of the development process. In addition to that, we also had a couple of issues creating a meeting time that suited everyone so that no one would have to compromise personal time. In the event that we did not use our pre-determined timeslot from the beginning of the year, it became difficult to find times to meet. 



\section{What Would You Do Differently Next Time}

Although the Utrition team is satisfied with the final product, certain goals we originally had set out for Utrition were not able to be met because of the time frame and resources that we had set aside for it. In order to be economic, the requirement we had set was that the logged data would be stored locally on the user's endpoint and the application would be run locally. Given the opportunity of creating another application or to continue building Utrition, storing the user's data on an external database and hosting the web application online would be a great improvement to the application.\\

In terms of future endeavours, the Utrition team has learned many key managing tips to ensure a successful project. One such tip is to do more research regarding implementation before creating project requirements. If our team had done further investigation into what was possible implementation-wise, we could have considered additional input methods rather than solely image upload. This would have saved us lots of time and effort that was used to revise our project goals. 

\end{document}