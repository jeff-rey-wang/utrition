\documentclass{article}

\usepackage{tabularx}
\usepackage{booktabs}

\title{Problem Statement and Goals\\\progname}

\author{\authname}

\date{}

\input{../Comments}
%% Common Parts

\newcommand{\progname}{Utrition} % PUT YOUR PROGRAM NAME HERE
\newcommand{\authname}{Team 16, Durum Wheat Semolina
	\\ Alexander Moica
	\\ Yasmine Jolly
	\\ Jeffrey Wang
	\\ Jack Theriault
	\\ Catherine Chen
	\\ Justina Srebrnjak } % AUTHOR NAMES                 

\usepackage{hyperref}
    \hypersetup{colorlinks=true, linkcolor=blue, citecolor=blue, filecolor=blue,
                urlcolor=blue, unicode=false}
    \urlstyle{same}
                                


\begin{document}

\maketitle

\begin{table}[hp]
\caption{Revision History} \label{TblRevisionHistory}
\begin{tabularx}{\textwidth}{llX}
\toprule
\textbf{Date} & \textbf{Developer(s)} & \textbf{Change}\\
\midrule
09/19/2022 & Catherine Chen, Yasmine Jolly, &Initial Document\\ 
&Jeffrey Wang, Jack Theriault, &\\
&Alex Moica, Justina Srebrnjak &\\
%Date2 & Name(s) & Description of changes\\
%... & ... & ...\\
\bottomrule
\end{tabularx}
\end{table}

\section{Problem Statement}

\wss{You should check your problem statement with the
\href{https://github.com/smiths/capTemplate/blob/main/docs/Checklists/ProbState-Checklist.pdf}
{problem statement checklist}.}
\wss{You can change the section headings, as long as you include the required information.}

\subsection{Problem}

\subsection{Inputs and Outputs}
\textbf{Inputs:}\\
The user will be able to upload either one or multiple images of one individual 
food item.\\
\textbf{Outputs:}\\
The application will accurately identify what food or foods the user has 
uploaded an image of. Additionally the application, upon identifying what food 
the user has uploaded an image of, will generate the nutrition facts and 
calories of this item.

\subsection{Stakeholders}

\textbf{Primary Stakeholders - } 

Health Conscious Individuals: The primary stakeholder for this web application will be health conscious individuals that wish to improve their daily eating habits. Assuming the user does not have the knowledge of nutritional values of foods they are estimated to utilize the app whenever they want to eat a meal approximately three to four times a day.

\textbf{Secondary Stakeholders - } 

Individuals Employed in the Health and Fitness Sector and Active Gym Users: Rather than relying on the application, it will be used more as a helpful tool to be used on occasion. The user is estimated to have a certain level of knowledge of nutrition. Rather than utilizing the application every meal it will be used when the user is unsure or has forgotten the foods nutrition facts.

\textbf{Tertiary Stakeholders - } 

Grocery Stores: Grocery stores stock may be affected as a response to users changing food patterns due to an increase in knowledge of nutritional facts. There may be a preference towards certain healthier foods.

Restaurants: They may expect a decrease in customers as there may be a preference to cook from home. As the utrition app gives guidance as to how many calories you are consuming the users may prefer to know what they are consuming which they do not know at a restaurant as you are unable to discern the individual ingredients. The customers who still decide to go to the restaurants may prefer going for a healthier option on the menu.

Gyms: Upon gaining the knowledge of how many calories they are consuming, users of the utrition app may attend the gym more often to burn off extra calories that may exceed their recommended caloric intake. 


\subsection{Environment}

\wss{Hardware and software}

\section{Goals}

\section{Stretch Goals}

\wss{This section contains project goals that are not integral to the development of Utrition, but would be valuable extensions to develop once the aforementioned goals are met.}

\subsection{Web Hosting}

\wss{Users will be able to access the Utrition application online through a website hosted on the internet. This will increase the user base of Utrition by making Utrition more accessible and convenient to use. Once hosted online, the delay between passing an input to Utrition and receiving an output will be similar in length to the local implementation of Utrition within a small margin of error. }

\subsection{Logging Preset Meals}

\wss{Users will be able to create presets of their commonly eaten meals on Utrition and save them for future use. Once a preset is created, a user will be able to select it to log the information of the meal without needing to manually re-upload images. This feature will allow users to customize Utrition to fit their eating habits, while also reducing the amount of time it takes to log a meal and access its information.  }

\end{document}
