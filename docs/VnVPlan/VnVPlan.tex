\documentclass[12pt, titlepage]{article}

\usepackage{booktabs}
\usepackage{tabularx}
\usepackage{hyperref}

\hypersetup{
    colorlinks,
    citecolor=blue,
    filecolor=black,
    linkcolor=red,
    urlcolor=blue
}
\usepackage[round]{natbib}

%% Comments

\usepackage{color}

\newif\ifcomments\commentsfalse %displays comments
%\newif\ifcomments\commentsfalse %so that comments do not display

\ifcomments
\newcommand{\authornote}[3]{\textcolor{#1}{[#3 ---#2]}}
\newcommand{\todo}[1]{\textcolor{red}{[TODO: #1]}}
\else
\newcommand{\authornote}[3]{}
\newcommand{\todo}[1]{}
\fi

\newcommand{\wss}[1]{\authornote{blue}{SS}{#1}} 
\newcommand{\plt}[1]{\authornote{magenta}{TPLT}{#1}} %For explanation of the template
\newcommand{\an}[1]{\authornote{cyan}{Author}{#1}}

%% Common Parts

\newcommand{\progname}{Utrition} % PUT YOUR PROGRAM NAME HERE
\newcommand{\authname}{Team 16, Durum Wheat Semolina
	\\ Alexander Moica
	\\ Yasmine Jolly
	\\ Jeffrey Wang
	\\ Jack Theriault
	\\ Catherine Chen
	\\ Justina Srebrnjak } % AUTHOR NAMES                 

\usepackage{hyperref}
    \hypersetup{colorlinks=true, linkcolor=blue, citecolor=blue, filecolor=blue,
                urlcolor=blue, unicode=false}
    \urlstyle{same}
                                


\begin{document}

\title{Project Title: System Verification and Validation Plan for \progname{}} 
\author{Author Name}
\date{\today}
	
\maketitle

\pagenumbering{roman}

\section{Revision History}

\begin{tabularx}{\textwidth}{p{3cm}p{2cm}X}
\toprule {\bf Date} & {\bf Version} & {\bf Notes}\\
\midrule
Date 1 & 1.0 & Notes\\
Date 2 & 1.1 & Notes\\
\bottomrule
\end{tabularx}

\newpage

\tableofcontents

\listoftables
\wss{Remove this section if it isn't needed}

\listoffigures
\wss{Remove this section if it isn't needed}

\newpage

\section{Symbols, Abbreviations and Acronyms}

\renewcommand{\arraystretch}{1.2}
\begin{tabular}{l l} 
  \toprule		
  \textbf{symbol} & \textbf{description}\\
  \midrule 
  T & Test\\
  \bottomrule
\end{tabular}\\

\wss{symbols, abbreviations or acronyms -- you can simply reference the SRS
  \citep{SRS} tables, if appropriate}

\newpage

\pagenumbering{arabic}

This document ... \wss{provide an introductory blurb and roadmap of the
  Verification and Validation plan}

\section{General Information}

\subsection{Summary}

\wss{Say what software is being tested.  Give its name and a brief overview of
  its general functions.}

\subsection{Objectives}

\wss{State what is intended to be accomplished.  The objective will be around
  the qualities that are most important for your project.  You might have
  something like: ``build confidence in the software correctness,''
  ``demonstrate adequate usability.'' etc.  You won't list all of the qualities,
  just those that are most important.}

\subsection{Relevant Documentation}

\wss{Reference relevant documentation.  This will definitely include your SRS
  and your other project documents (MG, MIS, etc).  You can include these even
  before they are written, since by the time the project is done, they will be
  written.}

\citet{SRS}

\section{Plan}

\wss{Introduce this section.   You can provide a roadmap of the sections to
  come.}

\subsection{Verification and Validation Team}

\wss{You, your classmates and the course instructor.  Maybe your supervisor.
  You shoud do more than list names.  You should say what each person's role is
  for the project.  A table is a good way to summarize this information.}

\subsection{SRS Verification Plan}

\wss{List any approaches you intend to use for SRS verification.  This may just
  be ad hoc feedback from reviewers, like your classmates, or you may have
  something more rigorous/systematic in mind..}

\wss{Remember you have an SRS checklist}

\subsection{Design Verification Plan}

\wss{Plans for design verification}

\wss{The review will include reviews by your classmates}

\wss{Remember you have MG and MIS checklists}

\subsection{Implementation Verification Plan}

\wss{You should at least point to the tests listed in this document and the unit
  testing plan.}

\wss{In this section you would also give any details of any plans for static verification of
  the implementation.  Potential techniques include code walkthroughs, code
  inspection, static analyzers, etc.}

\subsection{Automated Testing and Verification Tools}

\wss{What tools are you using for automated testing.  Likely a unit testing
  framework and maybe a profiling tool, like ValGrind.  Other possible tools
  include a static analyzer, make, continuous integration tools, test coverage
  tools, etc.  Explain your plans for summarizing code coverage metrics.
  Linters are another important class of tools.  For the programming language
  you select, you should look at the available linters.  There may also be tools
  that verify that coding standards have been respected, like flake9 for
  Python.}

\wss{The details of this section will likely evolve as you get closer to the
  implementation.}

\subsection{Software Validation Plan}

\wss{If there is any external data that can be used for validation, you should
  point to it here.  If there are no plans for validation, you should state that
  here.}

\section{System Test Description}
	
\subsection{Tests for Functional Requirements}

\wss{Subsets of the tests may be in related, so this section is divided into
  different areas.  If there are no identifiable subsets for the tests, this
  level of document structure can be removed.}

\wss{Include a blurb here to explain why the subsections below
  cover the requirements.  References to the SRS would be good.}

\subsubsection{Area of Testing1}

\wss{It would be nice to have a blurb here to explain why the subsections below
  cover the requirements.  References to the SRS would be good.  If a section
  covers tests for input constraints, you should reference the data constraints
  table in the SRS.}
		
\paragraph{Title for Test}

\begin{enumerate}

\item{test-id1\\}

Control: Manual versus Automatic
					
Initial State: 
					
Input: 
					
Output: \wss{The expected result for the given inputs}

Test Case Derivation: \wss{Justify the expected value given in the Output field}
					
How test will be performed: 
					
\item{test-id2\\}

Control: Manual versus Automatic
					
Initial State: 
					
Input: 
					
Output: \wss{The expected result for the given inputs}

Test Case Derivation: \wss{Justify the expected value given in the Output field}

How test will be performed: 

\end{enumerate}

\subsubsection{Area of Testing2}

...

\subsection{Tests for Nonfunctional Requirements}

\wss{The nonfunctional requirements for accuracy will likely just reference the
  appropriate functional tests from above.  The test cases should mention
  reporting the relative error for these tests.}

\wss{Tests related to usability could include conducting a usability test and
  survey.}

\subsubsection{Look and Feel Testing}
		
%\paragraph{Title for Test}

\begin{enumerate}
\item{NFR1\\} 

Type: Functional, Dynamic, and Manual.

Initial State: A developer from Durum Wheat Semolina has launched the Utrition app.

Input/Condition: The developer measures the distance between the displayed user interface components.

Output/Result: The distance between user interface components surpasses 20 pixels.

How test will be performed: Each developer launches the Utrition app and opens the pixel measuring tool (https://www.rapidtables.com/web/tools/pixel-ruler.html). The developer will use the tool by viewing a page on Utrition, then clicking “Print Screen'' on their keyboard. Then, the developer pastes the screenshot into the tool. The developer clicks the two spots he would like to measure distance from. The developer measures the distances between the following components:	

\begin{enumerate}
	\item[$-$] On the main menu:
	\begin{itemize}
		\item Main Menu Button
		\item Upload Image Button
		\item Past Nutritional Data Button
		\item Utrition Logo
		\item The Edge of the Application
	\end{itemize}
	\item[$-$] When uploading an image:
	\begin{itemize}
		\item Utrition Logo
		\item “Add More” Button
		\item Loading Bar
		\item Loading Text
		\item Main Menu Button
		\item The Edge of the Application	
	\end{itemize}
	\item[$-$] When viewing nutritional data:
	\begin{itemize}
		\item Utrition Logo
		\item Nutrition Data Text
		\item Associated Nutrition Symbol
		\item Continue Button
		\item Main Menu Button
		\item Photo of Identified Food
		\item Name of Identified Food
		\item The Edge of the Application	
	\end{itemize}
	\item[$-$] When viewing past nutritional data:
	\begin{itemize}
		\item Utrition Logo
		\item Nutrition Data Text
		\item Associated Nutrition Symbol
		\item Next Page Button
		\item Previous Page Button
		\item Main Menu Button
		\item Column and Row Headers
		\item The Edge of the Application
	\end{itemize}
	\item[$-$] When viewing past nutritional data chart:
	\begin{itemize}
		\item Utrition Logo
		\item Chart Entry Dates
		\item The Chart
		\item Nutrition Data Text
		\item Associated Nutrition Symbol
		\item Next Page Button
		\item Previous Page Button
		\item Back Button
		\item Main Menu Button
		\item The Edge of the Application	
	\end{itemize}
	
\end{enumerate}
	
\end{enumerate}

\subsubsection{Usability and Humanity Testing}

%\paragraph{Title for Test}

\begin{enumerate}

\item{NFR2\\}

Type: Functional, Dynamic, and Manual.
					
Initial State: The user is viewing a page of the user interface (Upload Image, View Nutritional Data, View Past Nutritional Data, View Past Nutritional Data Chart)
					
Input/Condition: The user clicks on the “Menu” button located at the bottom of the user interface.
					
Output/Result: The user will be brought to the main menu screen.

How test will be performed: A developer of Durum Wheat Semolina will access every page listed in “Initial State” and will attempt to reach the main menu in 2 or less clicks from the respective page.

					
\item{NFR3\\}

Type: Functional, Dynamic, and Manual.
					
Initial State: The user has previously entered 5 items in Utrition, and is now viewing the main menu.
					
Input: The user clicks on the “View Past Nutritional Data” button located at the right of the user interface.
					
Output: The user views a list displaying all 5 food items with their respective details in: food name, calories, fat, sodium, proteins, carbohydrates, sugars, and date entered.
					
How test will be performed: A developer of Durum Wheat Semolina will open Utrition on their personal device and access the main menu. The developer will upload an image of different food items 5 times, with a minimum 1 minute time difference between each input. The uploaded images of the food items will be randomly selected by the developer in the utrition/test/testPhotos directory. The developer closes Utrition and reopens it. The developer clicks on the “View Past Nutritional Data” button, and views the list displaying the 5 different food items and their respective details.

\item{NFR4\\}

Type: Functional, Dynamic, and Manual.

Initial State: The user has previously entered 5 items in Utrition, and has just clicked on the “View Past Nutritional Data” button.

Input/Condition: The user clicks on the “View Past Nutritional Data Chart” button located at the top of the user interface.

Output/Result: The user views a chart displaying all 5 food items with their respective details in: food name, calories, proteins, carbohydrates, sugars, and date entered.

How test will be performed: A developer of Durum Wheat Semolina will open Utrition on their personal device and access the main menu. The developer will upload an image of different food items 5 times, with a minimum 1 minute time difference between each input. The uploaded images of the food items will be randomly selected by the developer in the utrition/test/testPhotos directory. The developer closes Utrition and reopens it. The developer clicks on the “View Past Nutritional Data” button, and then clicks on the “View Past Nutritional Data Chart” button. The developer views a chart displaying 5 different food items and their respective details.

\item{NFR5\\}

Type: Functional, Dynamic, and Manual.

Initial State: The user views the main menu in Utrition without previously inputting any food items.

Input/Condition: The user clicks on the “View Past Nutritional Data” button located at the right of the user interface.

Output/Result: The user views an empty list with column entries: food name, calories, proteins, carbohydrates, sugars, and date entered.

How test will be performed: A developer of Durum Wheat Semolina will open Utrition on their personal device and access the main menu. The developer clicks on the “View Past Nutritional Data” button, and views the list displaying an empty list with the column entries listed in “Output”.

\item{NFR6\\}

Type: Functional, Dynamic, and Manual.

Initial State: The user has just clicked on the “View Past Nutritional Data” button without previously inputting any food items.

Input/Condition: The user clicks on the “View Past Nutritional Data Chart” button located at the top of the user interface.

Output/Result: The user views a chart displaying all 5 food items with their respective details in: food name, calories, proteins, carbohydrates, sugars, and date entered.

How test will be performed: A developer of Durum Wheat Semolina will open Utrition on their personal device and access the main menu. The developer clicks on the “View Past Nutritional Data” button, and then the “View Past Nutritional Data Chart” button. The developer views a chart displaying 5 different food items and their respective details.

\item{NFR7\\}

Type: Functional, Dynamic, and Manual.

Initial State: Utrition’s main menu is opened on the user’s personal device.

Input/Condition: The user clicks on the “Upload Image” button located at the middle of the main menu and uploads a photo of food.

Output/Result: The user views the food’s nutritional information.

How test will be performed: Each developer of Durum Wheat Semolina will load Utrition onto their personal device. Each developer will ask 2 people aged 14 or older to upload and retrieve the nutritional data of the food item “pineapple.png” found in the utrition/test/testPhotos directory. The developer will track the time it takes for each test.

\item{NFR8\\}

Type: Functional, Dynamic, and Manual.

Initial State: The user loads into Utrition’s main menu.

Input/Condition: The user views every page of the user interface (Upload Image, View Nutritional Data, View Past Nutritional Data, View Past Nutritional Data Chart).

Output/Result: The user will see the respective symbol associated with every mention of calories, fat, sodium carbohydrates, sugar, and protein.

How test will be performed: On each screen, including the main menu, a developer on Durum Wheat Semolina will check if there is a symbol associated with every mention of calories, fat, sodium carbohydrates, sugar, and protein. The developer will open Utrition on their personal device. The developer will click on the “Upload Image” button, and then the developer will upload a random image of food found in the testPhotos directory. The developer will view the nutritional information and then click on the “Menu” button. The developer clicks on the “View Past Nutritional Data” button, and then the “View Past Nutritional Data Chart” button.

\item{NFR9\\}

Type: Functional, Dynamic, and Manual.

Initial State: The user has previously entered 2 items in Utrition, and is now viewing the main menu.

Input/Condition: The user views every page of the user interface (Upload Image, View Nutritional Data, View Past Nutritional Data, View Past Nutritional Data Chart).

Output/Result: The user does not see any backend calculations.

How test will be performed: : On each screen, a developer on Durum Wheat Semolina will check if any backend calculations are displayed to the user. The developer will open Utrition on their personal device and access the main menu. The developer will upload an image of different food items 2 times, with a minimum 1 minute time difference between the inputs. The uploaded images of the food items will be randomly selected by the developer in the testPhotos directory. The developer closes Utrition and reopens it. The developer will click on the “Upload Image” button, and then the developer will upload an image of food. The developer will view the nutritional information and then click on the “Menu” button. The developer clicks on the “View Past Nutritional Data'' button, and then the “View Past Nutritional Data Chart” button.

\item{NFR10\\}

Type: Functional, Dynamic, and Manual.

Initial State: The user loads into Utrition’s main menu with a stable internet connection.

Input/Condition: The user views every page of the user interface (Upload Image, View Nutritional Data, View Past Nutritional Data, View Past Nutritional Data Chart).

Output/Result: 
\begin{itemize}
	\item Utrition is available to be used by users at all times since it is a web app.
	\item The user does not hear Utrition generate any sound.
\end{itemize}

How test will be performed: On each screen, a developer on Durum Wheat Semolina will check if a sound plays. The developer will open Utrition on their personal device. The developer will click on the “Upload Image” button, and then the developer will upload an image of food. The uploaded images of the food item will be randomly selected by the developer in the testPhotos directory The developer will view the nutritional information and then click on the “Menu” button. The developer clicks on the “View Past Nutritional Data” button, and then the “View Past Nutritional Data Chart” button.

\item{NFR11\\}

Type: Static and Manual.

Initial State: A developer on Durum Wheat Semolina views Utrition’s github page. 

Input/Condition: The developer checks every single directory if there is an audio file.

Output/Result: The developer removes all audio files found in Utrition’s github.
How test will be performed: The developer loads into Utrition’s github page (https://github.com/jeff-rey-wang/utrition/) and views the files in every single directory. For each directory the developer will check for the following file types:

\begin{itemize}
	\item MP3
	\item AAC
	\item Ogg Vorbis
	\item FLAC
	\item ALAC
	\item WAV
	\item AIFF
	\item DSD
	\item PCM
\end{itemize}
\end{enumerate}

\subsubsection{Performance Testing}

%\paragraph{Title for Test}

\begin{enumerate}
	
\item{NFR12\\} 

Type: Functional, Dynamic, and Manual.

Initial State: The user has previously entered 10 items in Utrition, and is now viewing the main menu.

Input/Condition: The user views every page of the user interface (Upload Image, View Nutritional Data, View Past Nutritional Data, View Past Nutritional Data Chart).

Output/Result: The user will be able to access every page of the user interface in 2 seconds or less.

How test will be performed: A developer on Durum Wheat Semolina will measure the time it takes for each new screen to load, excluding the screen that identifies each food item, using a stopwatch. The developer will open Utrition on their personal device and access the main menu. The developer will upload an image of different food items 10 times, with a minimum 1 minute time difference between the inputs. The uploaded images of the food items will be randomly selected by the developer in the utrition/test/testPhotos directory. The developer closes Utrition and reopens it. The developer will click on the “Upload Image” button, and then the developer will upload an image of food. The developer will view the nutritional information and then click on the “Menu” button. The developer clicks on the “View Past Nutritional Data'' button, and then the “View Past Nutritional Data Chart” button. The developer clicks on the “Menu” button.

\item{NFR13\\} 

Type: Functional, Dynamic, and Manual.

Initial State: The user loads into Utrition’s main menu, and has clicked on the “Upload Image” button.

Input/Condition: The user uploads 3 images simultaneously.

Output/Result: 
\begin{itemize}
	\item The user is able to view the identification for all 3 food items in 10 or less seconds.
	\item The user is able to view the nutritional information for all 3 food items in 5 or less seconds.
\end{itemize}

How test will be performed: A developer on Durum Wheat Semolina will open Utrition and click on the “Upload Image” button. The developer will upload a random image of a food item found in the testPhotos directory, and then click on the “Add More” button. The developer uploads 2 more random images of different food items and clicks to view the foods’ nutritional information. The developer measures the amount of time it takes for Utrition to notify the user the name of the identified food items. The developer measures the amount of time it takes for the system to change from the food identification interface to the nutritional information interface.

\item{NFR14\\} 

Type: Functional, Dynamic, and Manual.

Initial State: The user loads into Utrition’s main menu, and has clicked on the “Upload Image” 
button.

Input/Condition: The user uploads an image of food.

Output/Result: The food is correctly identified and the correlated nutritional facts are correct.
How test will be performed: A developer on Durum Wheat Semolina will open Utrition and click on the “Upload Image” button. The developer will upload the images of the following foods separately (images found in testPhotos directory):
\begin{itemize}
	\item Uncooked Pork
	\item Corn
	\item Lettuce
	\item Beef
	\item Penne Pasta
	\item Rice
	\item Milk
	\item Butter
	\item Wheat Bread
	\item Cooked Steak
\end{itemize} 

The developer will check the accuracy of each identified food item and their respective nutritional facts.

\item{NFR15\\} 

Type: Functional, Dynamic, and Manual.

Initial State: The user loads into Utrition’s main menu, and has clicked on the “Upload Image” button.

Input/Condition: The user uploads 4 photos of different food items.

Output/Result: The user is notified that they are not able to proceed with viewing the identified foods or their nutritional information.

How test will be performed: A developer on Durum Wheat Semolina will open Utrition and click on the “Upload Image” button. The developer will upload an image of a random food item found in the testPhotos directory, and then click on the “Add More” button. The developer uploads 3 more random images of different food. The developer is notified that they cannot proceed with viewing the nutritional information unless they remove 1 image.

\item{NFR16\\} 

Type: Static and Manual.

Initial State: A developer on Durum Wheat Semolina opens their internet browser.

Input/Condition: The developer is able to access \href{https://github.com/jeff-rey-wang/utrition}{Utrition’s github page} and download the repository onto their personal device.

Output/Result: Utrition is able to be downloaded by any user.

How test will be performed: The developer will access Utrition’s github on their personal device’s web browser. The developer will click on “Code”, and then “Download ZIP”. The developer will verify that Utrition has been downloaded onto their device.
\end{enumerate}

\subsubsection{Operational and Environmental Testing}

%\paragraph{Title for Test}

\begin{enumerate}

\item{NFR17\\} 

Type: Static and Manual.

Initial State: A user aged 14 or older has read Utrition’s “README.md” located at the bottom of the github page.

Input/Condition: The user installs Utrition on their personal device

Output/Result: The intended demographic is able to install Utrition on their computer.

How test will be performed: Each developer on Durum Wheat Semolina will send Utrition’s github to 5 of their peers. The developers will send the following message: “Hello! Whenever you have time available, could you please attempt to install my software? The installation guide can be found under “README.md” at the bottom of this page: \href{https://github.com/jeff-rey-wang/utrition/}{https://github.com/jeff-rey-wang/utrition/}. Please let me know if you were or were not able to install the program with the installation guide. Thanks!”

\item{NFR18\\} 

Type: Static and Manual.

Initial State: A user visits Utrition’s github page.

Input/Condition: The user checks that the latest change to Utrition’s github is not after April 30th, 2022.

Output/Result: Utrition will not have any updates after the final release.

How test will be performed: A developer on Durum Wheat Semolina will visit Utrition’s github page bi-yearly to ensure no updates are made to Utrition’s github.

\end{enumerate}

\subsubsection{Maintainability and Support Testing}

%\paragraph{Title for Test}

\begin{enumerate}

\item{NFR19\\} 

alex


\item{NFR20\\} 

Type: Functional, Dynamic, and Manual.

Initial State: The user has installed Utrition onto their Windows, macOS, or Linux device.

Input/Condition: The user is able to upload an image of food and view all pages of the user interface.

Output/Result: All of Utrition’s functionality can be accessed by Windows, macOS, and Linux devices

How test will be performed: Developers of Durum Wheat Semolina will install Utrition on 2 of each Windows, macOS, and Linux device. The developer will open Utrition on the device, and then clicks on the “Upload Image” button. The developer will upload a random image of a food item found in the testPhotos directory, and then click on the “Add More” button. The developer uploads 2 more random images of different food. The developer continues to see their foods’ nutritional information. The developer clicks on the “Menu” button. The developer clicks on the “View Past Nutritional Data” button, and then clicks on the “View Past Nutritional Data Chart” button.

\end{enumerate}

\subsubsection{Security Testing}

%\paragraph{Title for Test}

\begin{enumerate}

\item{NFR21\\} 

Type: Functional, Dynamic, and Manual.

Initial State: A user loads into Utrition’s main menu, and has clicked on the “Upload Image” button.

Input/Condition: The user uploads a PDF document as an image.

Output/Result: The user is notified that they are not able to proceed with viewing the identified foods or their nutritional information.

How test will be performed: A developer on Durum Wheat Semolina will open Utrition and click on the “Upload Image” button. The developer will upload the image “waffle.pdf” to Utrition. The developer is notified that they cannot proceed with viewing the nutritional information, since Utrition does not support “.pdf” file types.

\item{NFR22\\} 

Type: Static and Manual.

Initial State: A Durum Wheat Semolina developer visits Utrition’s github page.

Input/Condition: The developer checks the userdata directory for any files.

Output/Result: The userdata folder will be empty on github, so Utrition will only contain personalized data.

How test will be performed: The developer loads into \href{https://github.com/jeff-rey-wang/utrition/}{Utrition’s github page} and goes into the utrition/src/userdata directory. The developer ensures that no userdata is preloaded into Utrition’s github.


\end{enumerate}

\subsubsection{Cultural Testing}

%\paragraph{Title for Test}

\begin{enumerate}
\item{NFR23\\} 

Type: Functional, Dynamic, and Manual.

Initial State: Utrition’s main menu is being displayed to a user.

Input/Condition: The user is told by a Durum Wheat Semolina developer to find culturally insensitive material for a reward of 5 canadian dollars.

Output/Result: The user will be questioned after 5 minutes of searching if they have found any culturally insensitive material while using Utrition.

How test will be performed: Each developer will approach 5 peers with Utrition’s main menu open on the developer’s device. The developer will give full control over to their peers, and will tell their peers to search for culturally insensitive material for \$5. After 5 minutes are up, the developer will ask their peers if they have found any culturally insensitive material.

\end{enumerate}

\subsubsection{Legal Testing}

%\paragraph{Title for Test}

\begin{enumerate}
\item{NFR24\\}
 
Type: Static and Manual.
Initial State: A developer of Durum Wheat Semolina views Utrition’s github repository.
Input/Condition: The developer inspects all code to ensure Utrition is following Canada’s Anti-Spam Legislation Requirements for Installing Computer Programs, Canada’s Guide for Publishing Open Source Code, Google’s HTML/CSS Style Guide, and Google’s guide for material design.

Output/Result: Utrition is in compliance with all the standards laid out by the SRS document.

How test will be performed: The developer goes into the utrition/src directory, and then opens the link to one of the following guides:
\begin{itemize}
	\item \href{https://crtc.gc.ca/eng/internet/install.htm}{Canada’s Anti-Spam Legislation Requirements for Installing Computer Programs}
	\item \href{https://www.canada.ca/en/government/system/digital-government/digital-government-innovations/open-source-software/guide-for-publishing-open-source-code.html}{Guide for Publishing Open Source Code}
	\item \href{https://google.github.io/styleguide/jsguide.html}{Google JavaScript Style Guide}
	\item \href{https://google.github.io/styleguide/htmlcssguide.html}{Google HTML/CSS Style Guide}
	\item \href{https://material.io/design}{Material Design Guide}
\end{itemize}
The developer will go into each module found in the utrition/src directory, and will inspect the code to ensure that each rule outlined in Canada’s or Google’s guide is strictly adhered to. The developer is expected to review the code using the guide in chronological order.

\item{NFR25\\}

Type: Functional, Dynamic, and Manual.

Initial State: A Durum Wheat Semolina developer has completed a module for Utrition, and has Pylint installed on their device.

Input/Condition: The developer runs Pylint on their device.

Output/Result: Pylint will notify the developer if there needs to be changes to the code to ensure Utrition adheres to the PEP8 Python coding conventions.

How test will be performed: The developer will open their command prompt and travel to the directory with the finished module (utrition/src) by using “cd”. The developer runs Pylint on the finished module by typing “pylint finishedmodule.py” in their command prompt. The name “finishedmodule.py” is a placeholder for any future modules that Utrition will implement in the future. The command prompt will notify the developer to make any necessary changes.

	
\end{enumerate}
\wss{Reference your MIS and explain your overall philosophy for test case
  selection.}  
\wss{This section should not be filled in until after the MIS has
  been completed.}

\subsection{Unit Testing Scope}

\wss{What modules are outside of the scope.  If there are modules that are
  developed by someone else, then you would say here if you aren't planning on
  verifying them.  There may also be modules that are part of your software, but
  have a lower priority for verification than others.  If this is the case,
  explain your rationale for the ranking of module importance.}

\subsection{Tests for Functional Requirements}

\wss{Most of the verification will be through automated unit testing.  If
  appropriate specific modules can be verified by a non-testing based
  technique.  That can also be documented in this section.}

\subsubsection{Module 1}

\wss{Include a blurb here to explain why the subsections below cover the module.
  References to the MIS would be good.  You will want tests from a black box
  perspective and from a white box perspective.  Explain to the reader how the
  tests were selected.}

\begin{enumerate}

\item{test-id1\\}

Type: \wss{Functional, Dynamic, Manual, Automatic, Static etc. Most will
  be automatic}
					
Initial State: 
					
Input: 
					
Output: \wss{The expected result for the given inputs}

Test Case Derivation: \wss{Justify the expected value given in the Output field}

How test will be performed: 
					
\item{test-id2\\}

Type: \wss{Functional, Dynamic, Manual, Automatic, Static etc. Most will
  be automatic}
					
Initial State: 
					
Input: 
					
Output: \wss{The expected result for the given inputs}

Test Case Derivation: \wss{Justify the expected value given in the Output field}

How test will be performed: 

\item{...\\}
    
\end{enumerate}

\subsubsection{Module 2}

...

\subsection{Tests for Nonfunctional Requirements}

\wss{If there is a module that needs to be independently assessed for
  performance, those test cases can go here.  In some projects, planning for
  nonfunctional tests of units will not be that relevant.}

\wss{These tests may involve collecting performance data from previously
  mentioned functional tests.}

\subsubsection{Module ?}
		
\begin{enumerate}

\item{test-id1\\}

Type: \wss{Functional, Dynamic, Manual, Automatic, Static etc. Most will
  be automatic}
					
Initial State: 
					
Input/Condition: 
					
Output/Result: 
					
How test will be performed: 
					
\item{test-id2\\}

Type: Functional, Dynamic, Manual, Static etc.
					
Initial State: 
					
Input: 
					
Output: 
					
How test will be performed: 

\end{enumerate}

\subsubsection{Module ?}

...

\subsection{Traceability Between Test Cases and Modules}

\wss{Provide evidence that all of the modules have been considered.}
				
\bibliographystyle{plainnat}

\bibliography{../../refs/References}

\newpage

\section{Appendix}

This is where you can place additional information.

\subsection{Symbolic Parameters}

The definition of the test cases will call for SYMBOLIC\_CONSTANTS.
Their values are defined in this section for easy maintenance.

\subsection{Usability Survey Questions?}

\wss{This is a section that would be appropriate for some projects.}

\end{document}